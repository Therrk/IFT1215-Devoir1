\documentclass{article}
\usepackage[margin=1in]{geometry} 
\usepackage{amsmath,amsthm,amssymb,amsfonts, fancyhdr, color, comment, graphicx, environ}
\usepackage{xcolor}
\usepackage{mdframed}
\usepackage[shortlabels]{enumitem}
\usepackage{indentfirst}
\usepackage{hyperref}
\hypersetup{
    colorlinks=true,
    linkcolor=blue,
    filecolor=magenta,      
    urlcolor=blue,
}

\pagestyle{fancy}

\newenvironment{problem}[2][Problème]
    { \begin{mdframed}[backgroundcolor=gray!20] \textbf{#1 #2} \\}
    {  \end{mdframed}}

% Define solution environment
\newenvironment{solution}
    {\textit{Solution:}}
    {}

\renewcommand{\qed}{\quad\qedsymbol}

% prevent line break in inline mode
\binoppenalty=\maxdimen
\relpenalty=\maxdimen

%%%%%%%%%%%%%%%%%%%%%%%%%%%%%%%%%%%%%%%%%%%%%
%Fill in the appropriate information below
\lhead{Votre Nom: }
\rhead{IFT 1215} 
\chead{\textbf{Remise Devoir 1: 8 octobre 2021 }}
%%%%%%%%%%%%%%%%%%%%%%%%%%%%%%%%%%%%%%%%%%%%%

\begin{document}


\section{Conversion des Nombres}

\begin{problem}{1a)}
\textbf{[4pts]} Convertir $133_{10}$ en base 7 en utilisant la méthode de soustractions successifs, en binaire en utilisant la méthode de divisions successifs.
\end{problem}
\begin{solution}
    \begin{tabular}{c|c}
        Soustractions successives & Divisions successives\\
        $133 - 2\times(7^2) = 35$ & \\
        $35 - 5\times(7^1) = 0$ & \\
        $0 - 0\times(7^0) = 0$ & \\
        $133_{10} = 2\times(7^2) + 5\times(7^1) + 0\times(7^0)$ & \\
        $133_{10} = 250_7$ & 
    \end{tabular}
\end{solution}

\begin{problem}{1b)}
\textbf{[4pts]} Convertir $133_{10}$ en base 16, 8, 4 en passant par la base binaire
\end{problem}
\begin{solution}
    \begin{tabular}{cc|c|c}
    $133-2^7=5$ & Hexadécimal & Octal & Quaternaire\\
    $5-2^2 = 1$ & 1000 0101 & 010 000 101 & 10 00 01 01\\
    $1-2^0 = 0$ & 8 \hspace{14pt} 5 & 2 \hspace{6pt} 0 \hspace{6pt} 5 & 2 \hspace{4pt}0 \hspace{4pt}1 \hspace{4pt}1\\
    $133_{10} = 1000 0101_2$ & $85_{16}$ & $205_8$ & $2011_4$
    \end{tabular}
\end{solution}

\begin{problem}{1c)}
\textbf{[10pts]} Convertir $11001_2$ et $110001_2$ en décimal, octal et hexadécimal. Effectuer l’addition et la multiplication de ces nombres en binaire.
\end{problem}
\begin{solution}
    \begin{tabular}{r|r}
        \large Addition & \large multiplication\\
        \hline\\
        \huge $1^11000^11_2$  & \huge $110001_2$\\
        + \huge $11001_2$       & \huge $\times 11001_2$\\
        \rule{1in}{2pt}         & \rule{1in}{2pt}\\
        \huge $1001010_2$       &  \Large $1^110001_2$\\
         & \Large $0000000_2$ \hspace{6pt}\\
         & \Large $00000000_2$ \hspace{8pt}\\
         & \Large $110001000_2$ \hspace{12pt}\\
         & \Large $1100010000_2$ \hspace{12pt}\\
         & \rule{1.2in}{2pt}\\
         & \Large $10011001001_2$
    \end{tabular}
    \\

    \begin{tabular}{|c|c|c|c|c|c|}
        \hline
        \multicolumn{3}{|c|}{\large $11001_2$} & \multicolumn{3}{|c|}{\large $110001_2$}\\
        \hline
        Décimal & Octal & Hexadécimal & Décimal & Octal & Hexadécimal\\
        \hline
        00011001 & 011 001 & 0001 1001 & 00110001 & 110 001 & 0011 0001\\
        16+8+1& 3\hspace{10pt} 1 & 1 \hspace{5pt}9 & 32+16+1 & 6\hspace{10pt} 1 & 3\hspace{5pt} 1\\
        $25_{10}$ & $31_8$ & $19_{16}$ & $49_{10}$ & $61_8$ & $31_{16}$\\
        \hline
    \end{tabular}
\end{solution}

\begin{problem}{1d)}
\textbf{[4 pts]} Convertir $DAB_{16}$ en décimal et octal.
\end{problem}
\begin{solution}
    \begin{tabular}{c|c}
        décimal & Octal\\
        \hline
        D \hspace{40pt}A\hspace{40pt} B & D\hspace{15pt} A\hspace{15pt} B\\
        $13\times16^2 + 10\times16^1 + 11\times16^0$ & 1101 1010 1011\\
        $3328+160+11$ & $110$ $110$ $101$ $011$\\
        $3499$ & $6653$
        
    \end{tabular}
\end{solution}

\begin{problem}{1e)}
\textbf{[5pts]} Convertir les nombres $101.101_2$ et $10101.10001_2$ en octal et décimal. Pour la conversion en décimal, donnez votre réponse en forme de somme des puissances de 2.
\end{problem}
\begin{solution}
    \begin{tabular}{|c|c|c|c|}
        \hline
        \multicolumn{2}{|c|}{\large $101.101_2$} & \multicolumn{2}{|c|}{\large $10101.10001_2$}\\
        \hline
        Octal & Décimal & Octal & Décimal\\
        \hline
        101 . 101 & $101.101_2$ & 010 101 . 100 010 & 10101.10001 \\
        5  .  5 & $2^2+2^0+2^{-1}+2^{-3}$ & 2 \hspace{12pt}5 . 4\hspace{12pt}2 & $2^4+2^2+2^0+2^{-1}+2^{-5}$  \\
        $5.5_8$&$5.625_{10}$&$25.42_8$&$21.53125_{10}$\\
        \hline
    \end{tabular}
\end{solution}

\begin{problem}{1f)}
\textbf{[5 pts]} Convertir les nombres $11.625_{10}$ et $1/11_{10}$ en binaire. Si les fractions sont infinies et ne sont pas périodiques, cherchez les 6 chiffres après la virgule
\end{problem}
\begin{solution}
\end{solution}

\section{Format des Données Numériques}

\begin{problem}{2a)}
\textbf{[6 pts]} Représenter les nombres -1710 et 1710 selon les 3 conventions:
    \begin{itemize}
        \item Valeur Signée sur 8 bits
        \item DCB (décimal codé binaire) sur 16 bits
        \item Complément à deux sur 8 bits
    \end{itemize}
\end{problem}

\begin{solution}
    \begin{tabular}{c|c}
        $-1710$ & $1710$\\
        \hline
        $-1710+2^{10} = -686$ & $1710-2^{10}=686$\\
        $-686+2^9=-174$ & $686-2^9=174$\\
        $-174+2^7=-36$ & $174-2^7=36$\\
        $-36+2^5=-4$ & $36-2^5=4$\\
        $-4+2^2=0$ & $4-2^2=0$\\
        impossible, trop de bits?
    \end{tabular}
\end{solution}

\newpage
\begin{problem}{2b)}
\textbf{[10 pts]} Calculez les expressions suivantes exprimées en complément à deux en indiquant
celles pour lesquelles apparaît un débordement ou (et) une retenue(Carry).
\begin{itemize}
        \item 11011011 – 01001011 (calcul sur 8 bits)
        \item 1001 – 0011 (calcul sur 4 bits)
        \item 11101011 – 11011011 (calcul sur 8 bits)
        \item 11111010 + 11110111 (calcul sur 8 bits)
        \item 10011101 + 00000111 (calcul sur 8 bits)
    \end{itemize}
\end{problem}
\begin{solution}
    








\end{solution}

\begin{problem}{2c)}
\textbf{[8pts]} Représenter les nombres $-14.625_{10}$ et $8/11_{10}$ dans le format IEEE-754 précision simple
\end{problem}

\begin{solution}
\end{solution}


\begin{problem}{2d)}
\textbf{[12pts]} Chercher les valeurs décimales représentées par les chaînes en hexadécimal
suivantes. L’encodage utilisé est le format IEEE-754 simple précision.
\begin{itemize}
    \item A23BB000
    \item 33DD1000
    \item FF200000
\end{itemize}
\end{problem}

\begin{solution}
\end{solution}
\end{document}

