\documentclass{article}
\usepackage[margin=1in]{geometry} 
\usepackage{amsmath,amsthm,amssymb,amsfonts, fancyhdr, color, comment, graphicx, environ}
\usepackage{xcolor}
\usepackage{mdframed}
\usepackage[shortlabels]{enumitem}
\usepackage{indentfirst}
\usepackage{hyperref}
\usepackage{listings}
\usepackage{color}

\definecolor{dkgreen}{rgb}{0,0.6,0}
\definecolor{gray}{rgb}{0.5,0.5,0.5}
\definecolor{mauve}{rgb}{0.58,0,0.82}

\lstset{frame=tb,
  language=Python,
  aboveskip=3mm,
  belowskip=3mm,
  showstringspaces=false,
  columns=flexible,
  basicstyle={\small\ttfamily},
  numbers=none,
  numberstyle=\tiny\color{gray},
  keywordstyle=\color{blue},
  commentstyle=\color{dkgreen},
  stringstyle=\color{mauve},
  breaklines=true,
  breakatwhitespace=true,
  tabsize=3
}

\hypersetup{
    colorlinks=true,
    linkcolor=blue,
    filecolor=magenta,      
    urlcolor=blue,
}

\pagestyle{fancy}

\newenvironment{problem}[2][Problème]
    { \begin{mdframed}[backgroundcolor=gray!20] \textbf{#1 #2} \\}
    {  \end{mdframed}}

% Define solution environment
\newenvironment{solution}
    {\textit{Solution:}}
    {}

\renewcommand{\qed}{\quad\qedsymbol}

% prevent line break in inline mode
\binoppenalty=\maxdimen
\relpenalty=\maxdimen

%%%%%%%%%%%%%%%%%%%%%%%%%%%%%%%%%%%%%%%%%%%%%
%Fill in the appropriate information below
\lhead{Votre Nom: Élie Leblanc, Alban Guyon, Mathis Morra Fischer}
\rhead{IFT 1215} 
\chead{\textbf{Remise Devoir 1: 8 octobre 2021 }}
%%%%%%%%%%%%%%%%%%%%%%%%%%%%%%%%%%%%%%%%%%%%%

\begin{document}


\section{Conversion des Nombres}

\begin{problem}{1a)}
\textbf{[4pts]} Convertir $133_{10}$ en base 7 en utilisant la méthode de soustractions successifs, en binaire en utilisant la méthode de divisions successifs.
\end{problem}
\begin{solution}
    \begin{tabular}{c|c}
        Soustractions successives & Divisions successives\\
        $133 - 2\times(7^2) = 35$ & $133 = (66\times2)+1$\\
        $35 - 5\times(7^1) = 0$ & $66 = (33\times2)+0$\\
        $0 - 0\times(7^0) = 0$ & $33 = (16\times2) +1$\\
        $133_{10} = 2\times(7^2) + 5\times(7^1) + 0\times(7^0)$ & $16 = (8\times2) +0$\\
        $133_{10} = 250_7$ & $8 = (4\times2) +0$\\
        & $4=(2\times2)+0$\\
        & $2 = (1\times2)+0$\\
        & $1= (0\times2) +1$\\
        & $133_10 = 10000101_2$ 
    \end{tabular}
\end{solution}

\begin{problem}{1b)}
\textbf{[4pts]} Convertir $133_{10}$ en base 16, 8, 4 en passant par la base binaire
\end{problem}
\begin{solution}
    \begin{tabular}{cc|c|c}
    $133-2^7=5$ & Hexadécimal & Octal & Quaternaire\\
    $5-2^2 = 1$ & 1000 0101 & 010 000 101 & 10 00 01 01\\
    $1-2^0 = 0$ & 8 \hspace{14pt} 5 & 2 \hspace{6pt} 0 \hspace{6pt} 5 & 2 \hspace{4pt}0 \hspace{4pt}1 \hspace{4pt}1\\
    $133_{10} = 1000 0101_2$ & $85_{16}$ & $205_8$ & $2011_4$
    \end{tabular}
\end{solution}

\begin{problem}{1c)}
\textbf{[10pts]} Convertir $11001_2$ et $110001_2$ en décimal, octal et hexadécimal. Effectuer l’addition et la multiplication de ces nombres en binaire.
\end{problem}
\begin{solution}
    \begin{tabular}{r|r}
        \large Addition & \large multiplication\\
        \hline \\ 
        \huge $1^11000^11_2$  & \huge $110001_2$\\
        \huge $+ 11001_2$       & \huge $\times 11001_2$\\
        \rule{1in}{2pt}         & \rule{1in}{2pt}\\
        \huge $1001010_2$       &  \Large $1^110001_2$\\
         & \Large $0000000_2$ \hspace{6pt}\\
         & \Large $00000000_2$ \hspace{8pt}\\
         & \Large $110001000_2$ \hspace{12pt}\\
         & \Large $1100010000_2$ \hspace{12pt}\\
         & \rule{1.2in}{2pt}\\
         & \Large $10011001001_2$
    \end{tabular}
    \\

    \begin{tabular}{|c|c|c|c|c|c|}
        \hline
        \multicolumn{3}{|c|}{\large $11001_2$} & \multicolumn{3}{|c|}{\large $110001_2$}\\
        \hline
        Décimal & Octal & Hexadécimal & Décimal & Octal & Hexadécimal\\
        \hline
        00011001 & 011 001 & 0001 1001 & 00110001 & 110 001 & 0011 0001\\
        16+8+1& 3\hspace{10pt} 1 & 1 \hspace{5pt}9 & 32+16+1 & 6\hspace{10pt} 1 & 3\hspace{5pt} 1\\
        $25_{10}$ & $31_8$ & $19_{16}$ & $49_{10}$ & $61_8$ & $31_{16}$\\
        \hline
    \end{tabular}
\end{solution}
\newpage
\begin{problem}{1d)}
\textbf{[4 pts]} Convertir $DAB_{16}$ en décimal et octal.
\end{problem}
\begin{solution}
    \begin{tabular}{c|c}
        décimal & Octal\\
        \hline
        D \hspace{40pt}A\hspace{40pt} B & D\hspace{15pt} A\hspace{15pt} B\\
        $13\times16^2 + 10\times16^1 + 11\times16^0$ & 1101 1010 1011\\
        $3328+160+11$ & $110$ $110$ $101$ $011$\\
        $3499$ & $6653$
        
    \end{tabular}
\end{solution}

\begin{problem}{1e)}
\textbf{[5pts]} Convertir les nombres $101.101_2$ et $10101.10001_2$ en octal et décimal. Pour la conversion en décimal, donnez votre réponse en forme de somme des puissances de 2.
\end{problem}
\begin{solution}
    \begin{tabular}{|c|c|c|c|}
        \hline
        \multicolumn{2}{|c|}{\large $101.101_2$} & \multicolumn{2}{|c|}{\large $10101.10001_2$}\\
        \hline
        Octal & Décimal & Octal & Décimal\\
        \hline
        101 . 101 & $101.101_2$ & 010 101 . 100 010 & 10101.10001 \\
        5  .  5 & $2^2+2^0+2^{-1}+2^{-3}$ & 2 \hspace{12pt}5 . 4\hspace{12pt}2 & $2^4+2^2+2^0+2^{-1}+2^{-5}$  \\
        $5.5_8$&$5.625_{10}$&$25.42_8$&$21.53125_{10}$\\
        \hline
    \end{tabular}
\end{solution}

\begin{problem}{1f)}
    \textbf{[5 pts]} Convertir les nombres $11.625_{10}$ et $1/11_{10}$ en binaire. Si les fractions sont infinies et ne sont pas périodiques, cherchez les 6 chiffres après la virgule
\end{problem}
\begin{solution}
    \begin{tabular}{|c|c|}
        \hline
        $11.625_{10}$               &  \\
        $-8_{10}$                   & $2^3$ \\
        \hline          
        $=3.625_{10}$               &  \\
        $-2_{10}$                   & $2^1$ \\
        \hline          
        $=1.625_{10}$               &  \\
        $-1_{10}$                   & $2^0$ \\
        \hline          
        $=0.625_{10}$               &  \\
        $-0.5_{10}$                 & $2^{-1}$ \\
        \hline          
        $=0.125_{10}$               &  \\
        $-0.125_{10}$               & $2^{-3}$ \\
        \hline          
        =0                          & $1011.101_2$\\
        \hline
    \end{tabular}
    \\\\\\\\
    \begin{tabular}{|c|c|}
        \hline
        $1/11_{10}$                 & \\
        $-0.5$                      & $2^{-1}$ \\
        \hline
        $=0.4090909090\dots_{10}$   & \\
        $-0.25_{10}$                & $2^{-2}$ \\
        \hline
        $=0.1590909090\dots_{10}$   & \\
        $-0.125_{10}$               & $2^{-3}$ \\
        \hline
        $=0.0340909090\dots_{10}$   & \\
        $-0.3125_{10}$              & $2^{-5}$ \\
        \hline
        $=0.0028409090\dots_{10}$   & \\
        $-0.001953125{10}$          & $2^{-9}$ \\
        \hline
        $=0.0008877840\dots_{10}$   & $=0.111010001\dots_2$ \\
        \hline
                                    & $\approx0.111010_2$ \\
        \hline

    \end{tabular}
\end{solution}

\newpage

\section{Format des Données Numériques}

\begin{problem}{2a)}
\textbf{[6 pts]} Représenter les nombres -17 et 17 selon les 3 conventions:
    \begin{itemize}
        \item Valeur Signée sur 8 bits
        \item DCB (décimal codé binaire) sur 16 bits
        \item Complément à deux sur 8 bits
    \end{itemize}
\end{problem}

\begin{solution}
    \begin{tabular}{c|c}
        $-17$ & $17$\\
        \hline
        $-17+2^4 = -1$ & $17-2^4 = 1$\\
        $-1+2^0=0$ & $1-2^0=0$\\
        $10010001_2$ & $00010001_2$
    \end{tabular}
    \\
    \begin{tabular}[20pt]{c|c}
        $-17$ & $17$\\
        \hline
        nég \hspace{12pt}0 \hspace{12pt}1 \hspace{12pt} 7 \hspace{12pt}&pos \hspace{12pt}0 \hspace{12pt}1 \hspace{12pt} 7\\
        $1101$ $0000$ $0001$ $0111_2$ &$1011$ $0000$ $0001$ $0111_2$\\
        $110100010111_2$& $101100000010111_2$
    \end{tabular}
    \newline
    \begin{tabular}{c|c}
        $-17$ & $17$\\
        \hline
        $11111111_2 - 00010001_2$ & $00010001_2$\\
        $11101110_2 + 1_2$ & La représentation en complément à deux est pareille à la \\
        $11101111_2$ &  représentation à valeur signée pour les nombres positifs
    \end{tabular}
\end{solution}


\begin{problem}{2b)}
\textbf{[10 pts]} Calculez les expressions suivantes exprimées en complément à deux en indiquant
celles pour lesquelles apparaît un débordement ou (et) une retenue(Carry).
\begin{itemize}
        \item 11011011 – 01001011 (calcul sur 8 bits)
        \item 1001 – 0011 (calcul sur 4 bits)
        \item 11101011 – 11011011 (calcul sur 8 bits)
        \item 11111010 + 11110111 (calcul sur 8 bits)
        \item 10011101 + 00000111 (calcul sur 8 bits)
    \end{itemize}
\end{problem}
\begin{solution}
    








\end{solution}

\begin{problem}{2c)}
\textbf{[8pts]} Représenter les nombres $-14.625_{10}$ et $8/11_{10}$ dans le format IEEE-754 précision simple
\end{problem}

\begin{solution}
    Avec le programme que j'ai codé en python ci dessous, on peux calculer la mantisse d'un nombre facilement:
    \begin{lstlisting}
        import math
        n = 8/11      # le nombre dont on veux trouver la mantisse
        power = 5       # la puissance avec laqelle on commense a comparer
        binary = ''     # la representation binaire de n
        mantisse = ''   # mantisse de n dans une representation en point flotante

        # 23 a cause des 23 bits de la mantisse -
        # le 1 qui est sousentendu dans la convention +
        # la puissance du premier chiffre significatif de n
        final_power = -23 -1 + math.floor(math.log2(n))

        # on trouve les bits un a un
        while power > final_power:
            if 2 ** power <= n:
                n -= 2 ** power
                binary += '1'
            else:
                binary += '0'
            # ajouter un point decimal avant 
            # de passer aux puissances negatives
            if power == 0:
                binary += '.'
            power -= 1

        # la mantisse est les 23 derniers chiffres de la representation binaire
        mantisse = binary.replace('.', '')[-23:]
        print(f"La representation binaire de {n}: " + binary)
        print(f"La mantisse de la representation flotante de {n}: " + mantisse)
    \end{lstlisting}
    Le programme nous donne le resultat suivant:
    \begin{lstlisting}
        La representation binaire de n: 000000.101110100010111010001011
        La mantisse de la representation flotante de n: 01110100010111010001011
    \end{lstlisting}
    L'exponent doit alors etre $-1_{10}$ \\
    Qui est egale a -127 + 64 + 32 + 16 + 8 ou $01111110_2$ \\
    Et le nombre a representer est positif
    Alors la representation flotante de 8/11 est: \\
    $0\ 01111110\ 01110100010111010001011_2$\\
    De la meme facon, on peux calculer la representation flotante de $-14.625_{10}$ \\
    Le programme nous donne le resultat suivant:
    \begin{lstlisting}
        La representation binaire de n: 001110.10100000000000000000
        La mantisse de la representation flotante de n: 11010100000000000000000
    \end{lstlisting}
    L'exponent doit alors etre $3_{10}$ \\
    Qui est egale a -127 + 128 + 2 ou $10000010_2$ \\
    Et le nombre a representer est positif \\
    Alors la representation flotante de -14.625 est: \\
    $1\ 10000010\ 11010100000000000000000_2$
\end{solution}


\begin{problem}{2d)}
\textbf{[12pts]} Chercher les valeurs décimales représentées par les chaînes en hexadécimal
suivantes. L’encodage utilisé est le format IEEE-754 simple précision.
\begin{itemize}
    \item A23BB000
    \item 33DD1000
    \item FF200000
\end{itemize}
\end{problem}

\begin{solution}
\end{solution}
\end{document}

